The o\-Freq project does maintain a coding format standard. This eases the transition and interaction between all the various contributors. Please read the standard and please follow it. If you don't like the standard, suggest that we change it, but please do not ignore it.

\hyperlink{header_comment}{Header Comment Standard} Header Comment Standard

\hyperlink{cpp_comments}{C\-P\-P File Comment Standard} .C\-P\-P File Comment Standard

\hyperlink{object_paradigm}{Object Naming Paradigm} Object Naming Paradigm

\hyperlink{qt_platform_code}{Qt Platform Specific Code} Qt Platform Specific Code \hypertarget{header_comment}{}\section{Header Comment Standard}\label{header_comment}
The header format is strictly controlled. This provides a common format so that all of us have the header file to use as a roadmap when we read your code.\hypertarget{header_comment_requiredtext}{}\subsection{Legally Required Text}\label{header_comment_requiredtext}
If you ignore everything else about this format standard, at least place the following text at the top of every header file. This is a legal protection to ensure we don't get a lawsuit. \begin{DoxyVerb}/*----------------------------------------*- C++ -*------------------------------------------------------------------*\
| O pen         | OpenSea: The Open Source Seakeeping Suite                                                           |
| S eakeeping   | Web:     www.opensea.dmsonline.us                                                                   |
| E valuation   |                                                                                                     |
| A nalysis     |                                                                                                     |
\*------------------------------------------------------------------------------------------------------------------- \end{DoxyVerb}
 \hypertarget{cpp_comments}{}\section{C\-P\-P File Comment Standard}\label{cpp_comments}
The formatting of C\-P\-P files is mostly free. Organize the files any way that is most efficient for you. For the sake of your fellow coders, please make sure your file is organized by some logic. Best of all is to follow the organization logic of the header files.

\section*{Commenting of Files}

{\bfseries Heavily comment your files.}


\begin{DoxyItemize}
\item Include comments to describe every variable declaration
\item Comments should describe what you accomplish with each piece of code.
\begin{DoxyItemize}
\item Any time you perform any operation. Describe what that accomplished.
\end{DoxyItemize}
\item If you do anything that doesn't follow a straight linear sequence, explain it in the comments.
\item If you do something very clever, explain why you did it that way.
\item List any problems.
\item List any sections that need an error handler.
\end{DoxyItemize}

\section*{Required Text for Every .C\-P\-P File}

Place the following text at the top of every cpp file. This is a legal protection to ensure we don't get a lawsuit. Otherwise, comment styles are completely free within the file. Please comment heavily to explain your code. \begin{DoxyVerb}/*----------------------------------------*- C++ -*------------------------------------------------------------------*\
| O pen         | OpenSea: The Open Source Seakeeping Suite                                                           |
| S eakeeping   | Web:     www.opensea.dmsonline.us                                                                   |
| E valuation   |                                                                                                     |
| A nalysis     |                                                                                                     |
\*------------------------------------------------------------------------------------------------------------------- \end{DoxyVerb}
 \hypertarget{object_paradigm}{}\section{Object Naming Paradigm}\label{object_paradigm}
For ease of coding, we would like to have a single object naming paradigm. This controls how we name methods and functions contained within an object.\hypertarget{object_paradigm_objectname}{}\subsection{Object Names}\label{object_paradigm_objectname}
Individual object names and class names are fairly free and flexible. Just please make the naming convention logical. (And following a simple logic.)\hypertarget{object_paradigm_functionname}{}\subsection{Function Names}\label{object_paradigm_functionname}
Please use the following paradigm for naming any new functions in o\-Freq.


\begin{DoxyEnumerate}
\item Functions are described as an action prefix, followed by the subject, all as one word. Example\-: get\-Object
\item The action verb is not capitalized. The first leter of the subject is capitalized. If the subject is made of multiple words, write it all as one word, capitalizing each word. Example\-: get\-Object\-Two\-Because\-Its\-One\-Word
\item Use the following conventions for verbs on objects\-: get\-: Retrieve something from the object. Retrieved variable passed by value. ref\-: Retrieve something from the object. Retrieved variable passed by reference. set\-: Change some information into the object. Inserted variable passed by value. add\-: Insert some new information into the object. Usually associated with adding an entry to a vector. list\-: Access a vector from the object. list(index)\-: Access a specific item in the vector from the object.
\item Any time you create a list function, you must also create a list(index) function. 
\end{DoxyEnumerate}\hypertarget{qt_platform_code}{}\section{Qt Platform Specific Code}\label{qt_platform_code}
We have one set of source code, which compiles differently, depending on which platform gets selected. Bear this in mind when you write new source code. There are two methods in Qt to change source code at compile time, depending on platform.

\section*{Little Pieces of Code}

You can specify compiler options which depend on Qt system variables.

For little pieces of code you can simply use the following construct.

\begin{quotation}
\#if defined(\-Q\-\_\-\-O\-S\-\_\-\-W\-I\-N32) \#elif defined(\-Q\-\_\-\-O\-S\-\_\-\-M\-A\-C\-X) \#elif etc... \#endif

\end{quotation}


The variables you need are as follows (I included other O\-S's just for completeness, but we only use linux and windows\-:


\begin{DoxyEnumerate}
\item Q\-\_\-\-O\-S\-\_\-\-W\-I\-N32\-: Any Windows O\-S. Q\-\_\-\-O\-S\-\_\-\-L\-I\-N\-U\-X\-: Any linuux O\-S. Q\-\_\-\-O\-S\-\_\-\-U\-N\-I\-X\-: Unix O\-S. Q\-\_\-\-O\-S\-\_\-\-M\-A\-X\-: Mac O\-S.
\end{DoxyEnumerate}

All these macros are included as part of the $<$\-Qt\-Global$>$ header file.

\href{http://qt-project.org/doc/qt-5.0/qtcore/qtglobal.html}{\tt http\-://qt-\/project.\-org/doc/qt-\/5.\-0/qtcore/qtglobal.\-html}

\section*{Whole files}

You can also specify whole individual files to be included or excluded depending on which build is selected.

If you have bigger “implementation details” you can separate things in different cpp files one for each platform i.\-e. \-: mycoolwidget.\-cpp $<$-\/ common implementation mycoolwidget\-\_\-win.\-cpp $<$-\/ windows specific stuff mycoolwidget\-\_\-unix.\-cpp $<$-\/ linux/os x stuff if both can use the same code etc…

Then in your pro file, use scopes to build the correct set of files. If you need your own defines for a platform theres the D\-E\-F\-I\-N\-E\-S variable 
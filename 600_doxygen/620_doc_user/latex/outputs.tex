\begin{DoxyRefDesc}{Todo}
\item[\hyperlink{todo__todo000034}{Todo}]write documentation for outputs page.\end{DoxyRefDesc}


\section*{Calculation of R\-A\-O}

All outputs include the absolute magnitude of response. And the response amplitude operator (R\-A\-O) for that output. R\-A\-O's are their own list, given separately after the absolute response.

\section*{Basic Feedback Outputs}

\hyperlink{wave_directions}{Wave Directions}

\hyperlink{wave_frequencies}{Wave Frequencies}

\hyperlink{output_wave_spectra}{Wave Spectra}

\section*{Global Solution Outputs}

\hyperlink{global_derivative}{Global Derivative}

\hyperlink{global_motion}{Global Motion}

\hyperlink{global_velocity}{Global Velocity}

\hyperlink{global_acceleration}{Global Acceleration}

\section*{Local Solution Outputs}

\hyperlink{local_derivative}{Local Derivative}

\hyperlink{local_motion}{Local Motion}

\hyperlink{local_velocity}{Local Velocity}

\hyperlink{local_acceleration}{Local Acceleration}

\section*{Force Outputs}

\hyperlink{global_force}{Global Forces}

\hyperlink{local_force}{Local Forces}

\section*{Power Outputs}

Outputs about power are often used for research into items such as wave energy extraction devices. They may also be useful for preliminary assessments of body structures.

\hyperlink{power}{Power}

\section*{Efficiency Outputs}

These are some customized outputs used by ofreq for efficiency assessments of wave energy extraction devices.

\hyperlink{efficiency_absolute}{Absolute Efficiency}

\hyperlink{efficiency_relative}{Relative Efficiency}

\section*{Human Tolerance Outputs}

Sometimes, human tolerances are the limiting criteria for seakeeping performance. Ofreq includes outputs for classice algorithms on human tolerance, based on the ship motions.

\hyperlink{local_msi}{Motion Sickness Index}

\hyperlink{local_sma}{Subjective Motion Assessment} \hypertarget{wave_directions}{}\section{Wave Directions}\label{wave_directions}
\begin{DoxyRefDesc}{Todo}
\item[\hyperlink{todo__todo000031}{Todo}]write documentation for wave directions. \end{DoxyRefDesc}


\section*{Output}

\section*{Calculation Method}

\section*{Limitations}

\section*{Application to Custom Motion Models}

\section*{R\-A\-O Calculation}

State how the R\-A\-O is calculated. For every output, ofreq calculates both the true response value, and the R\-A\-O for that value. They are listed sequentially. First the response, then the R\-A\-O.

\subsection*{R\-A\-O Normalization.}

Sometimes it can be very confusing which part of the input wave properties are used to normalize the output. This section should explicitely state how the R\-A\-O was calculated.

\subsection*{R\-A\-O Units}

Sometimes it can be very confusing what the units of an R\-A\-O are. So this should clarify. \hypertarget{wave_frequencies}{}\section{Wave Frequencies}\label{wave_frequencies}
\begin{DoxyRefDesc}{Todo}
\item[\hyperlink{todo__todo000032}{Todo}]write documentation for wave frequencies. \end{DoxyRefDesc}


\section*{Output}

\section*{Calculation Method}

\section*{Limitations}

\section*{Application to Custom Motion Models}

\section*{R\-A\-O Calculation}

State how the R\-A\-O is calculated. For every output, ofreq calculates both the true response value, and the R\-A\-O for that value. They are listed sequentially. First the response, then the R\-A\-O.

\subsection*{R\-A\-O Normalization.}

Sometimes it can be very confusing which part of the input wave properties are used to normalize the output. This section should explicitely state how the R\-A\-O was calculated.

\subsection*{R\-A\-O Units}

Sometimes it can be very confusing what the units of an R\-A\-O are. So this should clarify. \hypertarget{output_wave_spectra}{}\section{Wave Spectra}\label{output_wave_spectra}
\begin{DoxyRefDesc}{Todo}
\item[\hyperlink{todo__todo000033}{Todo}]write documentation for wave spectra. This is feedback to the user to show exactly what magnitude of wave spectra ofreq calculated for each specified wave direction.\end{DoxyRefDesc}


\section*{Output}

\section*{Calculation Method}

\section*{Limitations}

\section*{Application to Custom Motion Models}

\section*{R\-A\-O Calculation}

State how the R\-A\-O is calculated. For every output, ofreq calculates both the true response value, and the R\-A\-O for that value. They are listed sequentially. First the response, then the R\-A\-O.

\subsection*{R\-A\-O Normalization.}

Sometimes it can be very confusing which part of the input wave properties are used to normalize the output. This section should explicitely state how the R\-A\-O was calculated.

\subsection*{R\-A\-O Units}

Sometimes it can be very confusing what the units of an R\-A\-O are. So this should clarify. \hypertarget{global_derivative}{}\section{Global Derivative}\label{global_derivative}
\begin{DoxyRefDesc}{Todo}
\item[\hyperlink{todo__todo000019}{Todo}]write documentation for global derivative.\end{DoxyRefDesc}


\section*{Output}

\section*{Calculation Method}

\section*{Limitations}

\section*{Application to Custom Motion Models}

\section*{R\-A\-O Calculation}

State how the R\-A\-O is calculated. For every output, ofreq calculates both the true response value, and the R\-A\-O for that value. They are listed sequentially. First the response, then the R\-A\-O.

\subsection*{R\-A\-O Normalization.}

Sometimes it can be very confusing which part of the input wave properties are used to normalize the output. This section should explicitely state how the R\-A\-O was calculated.

\subsection*{R\-A\-O Units}

Sometimes it can be very confusing what the units of an R\-A\-O are. So this should clarify. \hypertarget{global_motion}{}\section{Global Motion}\label{global_motion}
\begin{DoxyRefDesc}{Todo}
\item[\hyperlink{todo__todo000021}{Todo}]write documentation for global motions.\end{DoxyRefDesc}


\section*{Output}

\section*{Calculation Method}

\section*{Limitations}

\section*{Application to Custom Motion Models}

\section*{R\-A\-O Calculation}

State how the R\-A\-O is calculated. For every output, ofreq calculates both the true response value, and the R\-A\-O for that value. They are listed sequentially. First the response, then the R\-A\-O.

\subsection*{R\-A\-O Normalization.}

Sometimes it can be very confusing which part of the input wave properties are used to normalize the output. This section should explicitely state how the R\-A\-O was calculated.

\subsection*{R\-A\-O Units}

Sometimes it can be very confusing what the units of an R\-A\-O are. So this should clarify. \hypertarget{global_velocity}{}\section{Global Velocity}\label{global_velocity}
\begin{DoxyRefDesc}{Todo}
\item[\hyperlink{todo__todo000022}{Todo}]write documentation for global velocity.\end{DoxyRefDesc}


\section*{Output}

\section*{Calculation Method}

\section*{Limitations}

\section*{Application to Custom Motion Models}

\section*{R\-A\-O Calculation}

State how the R\-A\-O is calculated. For every output, ofreq calculates both the true response value, and the R\-A\-O for that value. They are listed sequentially. First the response, then the R\-A\-O.

\subsection*{R\-A\-O Normalization.}

Sometimes it can be very confusing which part of the input wave properties are used to normalize the output. This section should explicitely state how the R\-A\-O was calculated.

\subsection*{R\-A\-O Units}

Sometimes it can be very confusing what the units of an R\-A\-O are. So this should clarify. \hypertarget{global_acceleration}{}\section{Global Acceleration}\label{global_acceleration}
\begin{DoxyRefDesc}{Todo}
\item[\hyperlink{todo__todo000018}{Todo}]write documentation for global acceleration.\end{DoxyRefDesc}


\section*{Output}

\section*{Calculation Method}

\section*{Limitations}

\section*{Application to Custom Motion Models}

\section*{R\-A\-O Calculation}

State how the R\-A\-O is calculated. For every output, ofreq calculates both the true response value, and the R\-A\-O for that value. They are listed sequentially. First the response, then the R\-A\-O.

\subsection*{R\-A\-O Normalization.}

Sometimes it can be very confusing which part of the input wave properties are used to normalize the output. This section should explicitely state how the R\-A\-O was calculated.

\subsection*{R\-A\-O Units}

Sometimes it can be very confusing what the units of an R\-A\-O are. So this should clarify. \hypertarget{local_derivative}{}\section{Local Derivative}\label{local_derivative}
\begin{DoxyRefDesc}{Todo}
\item[\hyperlink{todo__todo000024}{Todo}]write documentation for local derivative.\end{DoxyRefDesc}


\section*{Output}

\section*{Calculation Method}

\section*{Limitations}

\section*{Application to Custom Motion Models}

\section*{R\-A\-O Calculation}

State how the R\-A\-O is calculated. For every output, ofreq calculates both the true response value, and the R\-A\-O for that value. They are listed sequentially. First the response, then the R\-A\-O.

\subsection*{R\-A\-O Normalization.}

Sometimes it can be very confusing which part of the input wave properties are used to normalize the output. This section should explicitely state how the R\-A\-O was calculated.

\subsection*{R\-A\-O Units}

Sometimes it can be very confusing what the units of an R\-A\-O are. So this should clarify. \hypertarget{local_motion}{}\section{Local Motion}\label{local_motion}
\begin{DoxyRefDesc}{Todo}
\item[\hyperlink{todo__todo000026}{Todo}]write documentation for local motion.\end{DoxyRefDesc}


\section*{Output}

\section*{Calculation Method}

\section*{Limitations}

\section*{Application to Custom Motion Models}

\section*{R\-A\-O Calculation}

State how the R\-A\-O is calculated. For every output, ofreq calculates both the true response value, and the R\-A\-O for that value. They are listed sequentially. First the response, then the R\-A\-O.

\subsection*{R\-A\-O Normalization.}

Sometimes it can be very confusing which part of the input wave properties are used to normalize the output. This section should explicitely state how the R\-A\-O was calculated.

\subsection*{R\-A\-O Units}

Sometimes it can be very confusing what the units of an R\-A\-O are. So this should clarify. \hypertarget{local_velocity}{}\section{Local Velocity}\label{local_velocity}
\begin{DoxyRefDesc}{Todo}
\item[\hyperlink{todo__todo000029}{Todo}]write documentation for local velocity.\end{DoxyRefDesc}


\section*{Output}

\section*{Calculation Method}

\section*{Limitations}

\section*{Application to Custom Motion Models}

\section*{R\-A\-O Calculation}

State how the R\-A\-O is calculated. For every output, ofreq calculates both the true response value, and the R\-A\-O for that value. They are listed sequentially. First the response, then the R\-A\-O.

\subsection*{R\-A\-O Normalization.}

Sometimes it can be very confusing which part of the input wave properties are used to normalize the output. This section should explicitely state how the R\-A\-O was calculated.

\subsection*{R\-A\-O Units}

Sometimes it can be very confusing what the units of an R\-A\-O are. So this should clarify. \hypertarget{local_acceleration}{}\section{Local Acceleration}\label{local_acceleration}
\begin{DoxyRefDesc}{Todo}
\item[\hyperlink{todo__todo000023}{Todo}]write documentation for local acceleration.\end{DoxyRefDesc}


\section*{Output}

\section*{Calculation Method}

This calculation should include the $ sin(\theta)$ terms. For example\-:

\[ \ddot x_2 = ... + g sin \left ( \theta \right ) \]

\section*{Limitations}

\section*{Application to Custom Motion Models}

\[ |I_2|=\left| \int_{0}^T \psi(t) \left\{ u(a,t)- \int_{\gamma(t)}^a \frac{d\theta}{k(\theta,t)} \int_{a}^\theta c(\xi)u_t(\xi,t)\,d\xi \right\} dt \right| \]

\section*{R\-A\-O Calculation}

State how the R\-A\-O is calculated. For every output, ofreq calculates both the true response value, and the R\-A\-O for that value. They are listed sequentially. First the response, then the R\-A\-O.

\subsection*{R\-A\-O Normalization.}

Sometimes it can be very confusing which part of the input wave properties are used to normalize the output. This section should explicitely state how the R\-A\-O was calculated.

\subsection*{R\-A\-O Units}

Sometimes it can be very confusing what the units of an R\-A\-O are. So this should clarify. \hypertarget{global_force}{}\section{Global Forces}\label{global_force}
\begin{DoxyRefDesc}{Todo}
\item[\hyperlink{todo__todo000020}{Todo}]write documentation for global forces\end{DoxyRefDesc}


\section*{Output}

\section*{Calculation Method}

\section*{Limitations}

\section*{Application to Custom Motion Models}

\section*{R\-A\-O Calculation}

State how the R\-A\-O is calculated. For every output, ofreq calculates both the true response value, and the R\-A\-O for that value. They are listed sequentially. First the response, then the R\-A\-O.

\subsection*{R\-A\-O Normalization.}

Sometimes it can be very confusing which part of the input wave properties are used to normalize the output. This section should explicitely state how the R\-A\-O was calculated.

\subsection*{R\-A\-O Units}

Sometimes it can be very confusing what the units of an R\-A\-O are. So this should clarify. \hypertarget{local_force}{}\section{Local Forces}\label{local_force}
\begin{DoxyRefDesc}{Todo}
\item[\hyperlink{todo__todo000025}{Todo}]write documentation for local forces\end{DoxyRefDesc}


\section*{Output}

\section*{Calculation Method}

\section*{Limitations}

\section*{Application to Custom Motion Models}

\section*{R\-A\-O Calculation}

State how the R\-A\-O is calculated. For every output, ofreq calculates both the true response value, and the R\-A\-O for that value. They are listed sequentially. First the response, then the R\-A\-O.

\subsection*{R\-A\-O Normalization.}

Sometimes it can be very confusing which part of the input wave properties are used to normalize the output. This section should explicitely state how the R\-A\-O was calculated.

\subsection*{R\-A\-O Units}

Sometimes it can be very confusing what the units of an R\-A\-O are. So this should clarify. \hypertarget{power}{}\section{Power}\label{power}
\begin{DoxyRefDesc}{Todo}
\item[\hyperlink{todo__todo000030}{Todo}]write documentation for Power. Power extracted should be based on each equation entered. We need a summary for each force. Possibly also a summary for total force. Remember that only a force, reactive in nature, can extract power.\end{DoxyRefDesc}


\section*{Output}

\section*{Calculation Method}

\section*{Limitations}

\section*{Application to Custom Motion Models}

\section*{R\-A\-O Calculation}

State how the R\-A\-O is calculated. For every output, ofreq calculates both the true response value, and the R\-A\-O for that value. They are listed sequentially. First the response, then the R\-A\-O.

\subsection*{R\-A\-O Normalization.}

Sometimes it can be very confusing which part of the input wave properties are used to normalize the output. This section should explicitely state how the R\-A\-O was calculated.

\subsection*{R\-A\-O Units}

Sometimes it can be very confusing what the units of an R\-A\-O are. So this should clarify. \hypertarget{efficiency_absolute}{}\section{Absolute Efficiency}\label{efficiency_absolute}
\begin{DoxyRefDesc}{Todo}
\item[\hyperlink{todo__todo000016}{Todo}]write documentation for Absolute Effciiency. Refer back to the Porpoise Buoy documentation. That should have some useful stuff on the various types of efficiency calculated.\end{DoxyRefDesc}


\section*{Output}

\section*{Calculation Method}

\section*{Limitations}

\section*{Application to Custom Motion Models}

\section*{R\-A\-O Calculation}

State how the R\-A\-O is calculated. For every output, ofreq calculates both the true response value, and the R\-A\-O for that value. They are listed sequentially. First the response, then the R\-A\-O.

\subsection*{R\-A\-O Normalization.}

Sometimes it can be very confusing which part of the input wave properties are used to normalize the output. This section should explicitely state how the R\-A\-O was calculated.

\subsection*{R\-A\-O Units}

Sometimes it can be very confusing what the units of an R\-A\-O are. So this should clarify. \hypertarget{efficiency_relative}{}\section{Relative Efficiency}\label{efficiency_relative}
\begin{DoxyRefDesc}{Todo}
\item[\hyperlink{todo__todo000017}{Todo}]write documentation for Relative Effciiency. Refer back to the Porpoise Buoy documentation. That should have some useful stuff on the various types of efficiency calculated.\end{DoxyRefDesc}


\section*{Output}

\section*{Calculation Method}

\section*{Limitations}

\section*{Application to Custom Motion Models}

\section*{R\-A\-O Calculation}

State how the R\-A\-O is calculated. For every output, ofreq calculates both the true response value, and the R\-A\-O for that value. They are listed sequentially. First the response, then the R\-A\-O.

\subsection*{R\-A\-O Normalization.}

Sometimes it can be very confusing which part of the input wave properties are used to normalize the output. This section should explicitely state how the R\-A\-O was calculated.

\subsection*{R\-A\-O Units}

Sometimes it can be very confusing what the units of an R\-A\-O are. So this should clarify. \hypertarget{local_msi}{}\section{Motion Sickness Index}\label{local_msi}
\begin{DoxyRefDesc}{Todo}
\item[\hyperlink{todo__todo000027}{Todo}]write documentation for Motion Sickness Index. There will probably also be some inputs to enter.\end{DoxyRefDesc}


\section*{Output}

\section*{Calculation Method}

\section*{Limitations}

\section*{Application to Custom Motion Models}

\section*{R\-A\-O Calculation}

State how the R\-A\-O is calculated. For every output, ofreq calculates both the true response value, and the R\-A\-O for that value. They are listed sequentially. First the response, then the R\-A\-O.

\subsection*{R\-A\-O Normalization.}

Sometimes it can be very confusing which part of the input wave properties are used to normalize the output. This section should explicitely state how the R\-A\-O was calculated.

\subsection*{R\-A\-O Units}

Sometimes it can be very confusing what the units of an R\-A\-O are. So this should clarify. \hypertarget{local_sma}{}\section{Subjective Motion Assessment}\label{local_sma}
\begin{DoxyRefDesc}{Todo}
\item[\hyperlink{todo__todo000028}{Todo}]write documentation for Subjective Motion Assessment. There will probably also be some inputs to enter.\end{DoxyRefDesc}


\section*{Output}

\section*{Calculation Method}

\section*{Limitations}

\section*{Application to Custom Motion Models}

\section*{R\-A\-O Calculation}

State how the R\-A\-O is calculated. For every output, ofreq calculates both the true response value, and the R\-A\-O for that value. They are listed sequentially. First the response, then the R\-A\-O.

\subsection*{R\-A\-O Normalization.}

Sometimes it can be very confusing which part of the input wave properties are used to normalize the output. This section should explicitely state how the R\-A\-O was calculated.

\subsection*{R\-A\-O Units}

Sometimes it can be very confusing what the units of an R\-A\-O are. So this should clarify. 